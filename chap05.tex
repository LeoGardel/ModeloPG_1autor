\chapter{Conclusão e Trabalhos Futuros}

\section{Conclusão}

O \textit{framework} Ruby on Rails proporcionou o desenvolvimento da aplicação de forma extremamente ágil. O tempo para produção desta poderia ser ainda mais reduzido, caso não fosse preciso um tempo de aprendizado para utilização de diversas ferramentas e metodologias. Porém, vale ressaltar que ambos os programadores do projeto possuíam razoável experiência na utilização deste \textit{framework}, tendo passado por um processo de aprendizado relativamente longo. Desta forma, este projeto razoavelmente complexo pôde ser produzido em um número reduzido de horas de trabalho.

Técnicas como o TDD contribuíram enormemente para a produção de um código claro e manutenível, por todas as razões elucidadas anteriormente. Foi percebida uma dificuldade inicial na adoção deste método, já que os desenvolvedores não estavam habituados à realização de testes de forma precoce. Entretanto, a sua utilização proporcionou uma experiência positiva, deixando claras as vantagens da utilização desta técnica.

Ao longo do curso de graduação, foi introduzido apenas o conceito de plano de negócio. Porém, após pesquisas, foi descoberto que o BMC é uma técnica mais atual e interessante para empresas iniciantes. O seu aprendizado e realização foi uma experiência extremamente valiosa, pois possibilita o empreendimento de forma ágil.

Pode-se concluir através dos resultados demonstrados pelo BMC que o projeto possui um bom potencial de rentabilidade. Desta forma, pretende-se dar continuidade a este empreendimento visando torná-lo público em breve. Entretanto, para que isso seja possível se fazem necessários alguns ajustes, que serão enumerados na seção a seguir.

\section{Trabalhos Futuros}

\subsection{Pagamento}

A funcionalidade de pagamento do sistema, apesar de implementada, não está otimizada, sendo necessárias melhorias:

\begin{itemize}
\item Transferência Automática de Verba para Organizadores

Como citado anteriormente, um gargalo para a escalabilidade do projeto é a necessidade de transferência manual de verbas para os organizadores dos eventos. Devido à dificuldade de implementação, esta característica foi deixada para o futuro.

\item Fraudes

Existe uma série de pontos de falha que deve ser cuidadosamente coberta. Um deles consiste na criação de eventos falsos com a finalidade de angariar fundos dos usuários por meio da funcionalidade de pagamento. A princípio, este problema pode ser sanado através de uma verificação presencial a respeito da existência do evento, porém, este método prejudica enormemente a escalabilidade do projeto. Diversas possíveis soluções foram pensadas para evitar estes e outros tipos de fraudes, porém todas possuem brechas, sendo necessária uma análise de risco mais aprofundada.

\item Sistema Próprio de Pagamento

A utilização de sistemas de pagamento terceirizados (como o PayPal) são muito custosos ao empreendimento. A complexidade da implementação de um sistema de pagamentos “online” inviabilizou a sua execução até o momento, porém, é sabido que este é um ponto chave para o sucesso do projeto.

\end{itemize}

\subsection{Segurança}

A aplicação ainda possui diversas brechas de segurança, que precisam ser identificadas e combatidas. É necessário um longo trabalho de aprimoramento até que tenhamos uma versão de fato segura.

\subsection{Parcerias e Sociedade}
\begin{itemize}

\item Promotores

Como já dito anteriormente, as parcerias ajudam no alavancamento do site. Então, para um crescimento acelerado, um dos próximos passos seria a busca por promotores interessados.

\item Designers

Uma parte que ainda está precária no sistema devido à falta de conhecimento por parte dos desenvolvedores, é a parte de design. Dessa forma, torna-se importante a parceria ou sociedade com algum designer ou empresa do ramo, visando a produção de um resultado ainda melhor.

\item Investidores

Outro foco que deve ser dado aos trabalhos futuros, é a busca por recursos financeiros, já que, apesar de baixos, o empreendimento tem custos que precisam ser sanados. Além disso, seria interessante que os criadores pudessem dar atenção total ao projeto, sendo necessário um salário para mantê-los.

\end{itemize}

Além destas, também são necessárias outras parcerias, como para gerência administrativa e de marketing do projeto. Porém estas só são necessárias em uma fase mais avançada do empreendimento, já que inicialmente não existe fluxo de caixa que torne-as indispensáveis.